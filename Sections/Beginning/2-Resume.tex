\vspace{-0.7cm}
Le réchauffement climatique est un des principaux défis de l'humanité. Les combustibles fossiles sont une importante source de production d'énergie dans le monde, entraînant des émissions de gaz à effet de serre (GES) dans l'atmosphère. L'Accord de Paris sur le climat a comme objectif principal de limiter la hausse de température à 2 \degree C, et si possible à 1,5 \degree C. Or, des rapports montrent que l'on se dirige vers une augmentation de 2,8 \degree C. Le secteur des centres de données est un important émetteur de GES, responsable de 1\% des émissions mondiales liées à l'énergie. Un moyen de réduire leur impact est de réduire l'énergie nécessaire aux serveurs. Mais, des auteurs signalent une réduction des améliorations énergétiques des processeurs, ce qui, lié à l'augmentation prévue du nombre d'internautes (jusqu'à 5,3 milliards en 2023), tend à aggraver la situation. Une solution serait de faire passer la production d'énergie des combustibles fossiles aux sources vertes (sources d'énergie renouvelables - SER), qui génèrent de l'énergie à partir de sources naturelles, comme l'énergie solaire et éolienne. Le principal inconvénient des SER est leur intermittence, car leur production dépend de la nature. Les grands fournisseurs d'informatique investissent dans la production renouvelable hors site pour fournir au réseau la même proportion d'énergie qu'ils consomment, transférant le problème de l'intermittence à des tiers. Cette thèse se situe dans le cadre du projet Datazero2 qui propose la création d'un centre de données avec une production d'énergie renouvelable sur site, ce qui réduirait considérablement les émissions de GES. Mais un centre de données 100\% renouvelable introduit des défis en matière d'énergie et de technologies de l'information. Pour les traiter, on les divise en deux parties : hors ligne et en ligne. La première planifie les décisions à partir des prévisions, prenant en compte le long terme. La partie en ligne applique le plan hors ligne et réagit aux événements réels. Comme les prévisions sont imparfaites, le plan en ligne doit trouver des moyens d'adapter le hors ligne, en réduisant l'impact sur les tâches des utilisateurs. En considérant ces éléments, cette thèse propose des approches pour combiner les décisions hors ligne et en ligne, en traitant l'incertitude de la production renouvelable et de la demande de la charge de travail. Tout d'abord, on propose quatre politiques de compensation énergétique dont l'objectif principal est d'adapter le plan prévu hors ligne pour rapprocher les niveaux de stockage des niveaux prévus à la fin d'une fenêtre temporelle. Ces adaptations sont nécessaires car la production et la demande d'énergie peuvent varier. On montre l'impact du respect des niveaux de stockage sur la qualité de service. Ensuite, on introduit des algorithmes d'apprentissage par renforcement dans le problème de la compensation afin d'améliorer la qualité de service de nos politiques. Ainsi, on utilise ces algorithmes pour définir la compensation des différents scénarios, dans le but d'augmenter le nombre de travaux terminés. En effet, on peut respecter les niveaux de stockage d'énergie tout en améliorant la qualité de service. Enfin, on présente l'heuristique \emph{\systemName}, qui combine les décisions relatives à la puissance et à l'ordonnancement pour réduire le nombre de tâches abandonnées et le gaspillage d'énergie. Elle utilise des prédictions pour prendre des décisions, en respectant les limites de la batterie pour prolonger sa vie. \emph{\systemName} respecte les contraintes de puissance, avec un faible gaspillage d'énergie et le plus faible nombre de tâches détruites parmi tous les algorithmes.
