\documentclass[12pt,a4paper,oneside]{book}
\usepackage[ED=MITT-MITT, Ets=INP]{tlsflyleaf}
%\usepackage[ED=SDU2E-Ast, ED2=SDU2E-Eco, Ets=UT3]{tlsflyleaf}
\usepackage[utf8]{inputenc}
\usepackage[T1]{fontenc}
\usepackage{helvet}
% \renewcommand{\familydefault}{\sfdefault}
\usepackage{hyperref}
\hypersetup{
    colorlinks,
    citecolor=black,
    filecolor=black,
    linkcolor=black,
    urlcolor=black
}
\usepackage{afterpage}


\setlength{\columnseprule}{0pt}
\setlength\columnsep{10pt}

% ==================
% Setup basic string
% - PhD Title
% - author
% - defence date
% - laboratory
% - cotutelle
\title{\textbf{\large On-line scheduling for IT tasks and power source commitment in datacenters only operated with renewable energy}}
\author{Igor FONTANA DE NARDIN}
\defencedate{31/01/2023}
\lab{Laplace (UMR 5213) et IRIT (UMR 5505)}
%\cotutelle{Nom de l'\'etablissement}

% ==================
% Setup people like your boss, the jury team and the referees
% - First you need to define how number they will be in each category
%   It is done with the commands \nboss{n}, \nreferee{n} and \njudge{n}.
%   You can define more people in each category than the number given 
%   but only the first "\npeople" will be print.
% - Then use the command \makesomeone{<category>}{<number>}{<name>}{<status>}{<other>}
%   where:
%     <category> should be selected in ['boss', 'referee', 'judge']
%     <number>   is the rank for printing the person. 
%                Only number <= "\npeople" will be printed
%     <name>     First name and last name of the people
%     <status>   Is (s)he a "charg\'e de recher" ou un "professeur d'universit\'e"...
%     <other>    Whatever string you want to add (laboratory, jury member place...).
%% Boss
\nboss{2}
\makesomeone{boss}{2}{Stéphane CAUX}{}{}  % Sera affiche en second
\makesomeone{boss}{1}{Patricia STOLF}{}{} % Sera afiche en premier
%% Referee
\nreferee{2}
\makesomeone{referee}{1}{Premier RAPPORTEUR}{}{}
\makesomeone{referee}{2}{Second RAPPORTEUR}{}{}
%% Judges
\njudge{5}
\makesomeone{judge}{1}{Premier MEMBRE}{Professeur d'Université}{Rapporteur}
\makesomeone{judge}{2}{Second MEMBRE}{Professeur d'Université}{Rapporteur}
\makesomeone{judge}{3}{Troisième MEMBRE}{Professeur d'Université}{Examinateur}
\makesomeone{judge}{4}{Quatrième MEMBRE}{Professeur d'Université}{Examinateur}
\makesomeone{judge}{5}{Cinquième MEMBRE}{Professeur d'Université}{Examinateur}

\graphicspath{{Images/}}

\newcommand\blankpage{%
    \null
    \thispagestyle{empty}%
    \addtocounter{page}{-1}%
    \newpage}


\usepackage{titlesec}
\titleformat{\chapter}[display]
  {\Huge\bfseries}
  {\chaptertitlename\ \thechapter}{20pt}{\Huge}

% ============================================================
% DOCUMENT
\begin{document}
    \makeflyleaf

    \afterpage{\blankpage}

    \pagenumbering{roman}
    Global warming is one of the biggest humanity challenges. Fossil fuels are one of the world's main sources of energy production, which emits greenhouse gas emissions (GHG) in the atmosphere. This gas creates a barrier in the atmosphere (like in a greenhouse), not allowing heat dissipation in the space. Therefore, Earth's temperature increases, unbalancing our environment. On 12 December 2015, the United Nations Climate Change Conference (COP21) established some objectives for the end of Century 21. The main goal is to limit the temperature to 2 \degree C while pursuing measures to limit it to 1.5 \degree C. However, recent reports show we are walking toward a temperature increase of 2.8 \degree C. High temperatures lead to several problems, such as heatwaves, droughts, and floods, impacting flora and fauna directly, increasing food and water insecurity, increasing mortality, impacting labor productivity, impairing learning, increasing adverse pregnancy outcomes possibility, increasing conflict, hate speech, migration, and infectious disease spread.

All industry and academic sectors are rethinking their processes and proposing solutions to the global warming problem. A significant GHG emitter is the Information and Communications Technology (ICT) sector, which englobes, for example, data centers, computers, and networks. Data centers alone are responsible for 1\% of the global energy-related GHG emissions. For example, Google data centers expended the same amount of energy as the entire city of San Francisco in 2015. A way to reduce the data center's impact on GHG emissions is by reducing the energy needed by the servers. However, some authors pointed out a reduction in the processor technologies improvements. Furthermore, with the expected increase in Internet users (up to 5.3 in 2023), the situation tends to get even worse. 

A possible solution is migrating energy production from brown sources (fossil fuels) to green sources (renewable energy sources - RES). It generates energy from natural sources, such as solar, wind, geothermal, hydropower, wave and tidal, and biomass. The renewable term comes from the idea that these sources are constantly replenished. The biggest drawback of implementing RES is its intermittence since its production comes from nature, depending on the climate conditions. Even so, big cloud providers, such as Google, Amazon, and Facebook, invest in solar and wind power plants. They implemented the off-site approach, which means they provide renewable energy to the grid with the same amount they expend. Therefore, they transfer the intermittence problem to third parties. 

Creating a renewable-only data center with only on-site renewable production would drastically reduce the GHG emissions by the data centers. This is what Datazero2 proposes. However, a renewable-only data center introduces several challenges in both power and IT parts. On the power side, a manager must decide the energy storage usage according to the estimated and real renewable production. On the other hand, the IT side must take the power production and execute the user's jobs. It is possible to divide the problem into two parts: offline and online. The offline side uses predictions to plan short-term decisions but considers the long-term. For example, the offline can estimate a power underproduction and use more energy from storage in the short-term but planning to recharge the storage shortly. On the online side, the manager must apply the offline plan, reacting to the actual events of the data center. Since the predictions are imperfect, the online must find ways to adapt the offline plan, reducing the impact on the user's jobs. Considering these elements, this thesis proposes some approaches to mix offline and online decisions, dealing with the uncertainty coming from renewable production and workload demand.

First, we propose four policies for energy compensation. The main objective of these policies is to adapt the offline plan to approximate the storage levels to the planned levels at the end of a time window. These adaptations are necessary since power production and demand can vary. For example, a renewable overproduction would produce more energy than predicted. So, the online can use this energy to run more jobs. On the other hand, it must reduce the usage in an underproduction scenario. Approximating the storage levels to the planned levels is crucial because energy storage is limited. Also, the data center runs uninterruptedly. So, it is not viable to always use more battery than expected for every time window. We demonstrate the impact of respecting the storage levels on the Quality of Service. 

After that, we introduce Reinforcement Learning algorithms in the compensation problem, trying to improve the Quality of Service of our policies. More specifically, we use these algorithms to define how to compensate for different scenarios, aiming to increase the number of finished jobs. Therefore, we could still respect the energy storage levels but also improve the Quality of Service. 

Finally, our last contribution is a heuristic that mixes both power and scheduling decisions, seeking to reduce the number of killed jobs and wasted energy. This heuristic is named \emph{\systemName} and uses predictions to make better decisions, finding dangerous moments to be more cautious. \emph{\systemName} respects the power constraints, having low wasted energy, and the lower number of killed jobs among all algorithms.
    \cleardoublepage
    Le réchauffement climatique est un des principaux défis de l'humanité. Les combustibles fossiles sont une importante source de production d'énergie dans le monde, entraînant des émissions de gaz à effet de serre (GES) dans l'atmosphère. Ces gaz y créent une barrière, empêchant la chaleur de se dissiper dans l'espace. Par conséquent, la température de la Terre augmente en déséquilibrant l'environnement. L'Accord de Paris sur le climat a comme objectif principal de limiter la hausse de température à 2 \degree C, et si possible à 1,5 \degree C. Or, des rapports montrent que l'on se dirige vers une augmentation de 2,8 \degree C. Les températures élevées entraînent des problèmes environnementaux et de santé. Différents secteurs repensent ainsi leurs processus et proposent des solutions au problème du réchauffement climatique. Le secteur des centres de données est un important émetteur de GES, responsable de 1\% des émissions mondiales liées à l'énergie. Un moyen de réduire leur impact est de réduire l'énergie nécessaire aux serveurs. Mais, des auteurs signalent une réduction des améliorations énergétiques des processeurs, ce qui, lié à l'augmentation prévue du nombre d'internautes (jusqu'à 5,3 milliards en 2023), tend à aggraver la situation. 

Une solution serait de faire passer la production d'énergie des combustibles fossiles aux sources vertes (sources d'énergie renouvelables - SER), qui génèrent de l'énergie à partir de sources naturelles, comme l'énergie solaire et éolienne. Le principal inconvénient des SER est leur intermittence, car leur production dépend de la nature. Les grands fournisseurs d'informatique investissent dans la production renouvelable hors site pour fournir au réseau la même proportion d'énergie qu'ils consomment, transférant le problème de l'intermittence à des tiers. Le projet Datazero2 propose la création d'un centre de données avec une production d'énergie renouvelable sur site, ce qui réduirait considérablement les émissions de GES. Mais un centre de données 100\% renouvelable introduit des défis en matière d'énergie et de technologies de l'information. Pour les traiter, on les divise en deux parties : hors ligne et en ligne. La première planifie les décisions à court terme à partir des prévisions, prenant en compte le long terme. La partie en ligne applique le plan hors ligne et réagit aux événements réels. Comme les prévisions sont imparfaites, le plan en ligne doit trouver des moyens d'adapter le hors ligne, en réduisant l'impact sur les tâches des utilisateurs. En considérant ces éléments, cette thèse propose des approches pour combiner les décisions hors ligne et en ligne, en traitant l'incertitude de la production renouvelable et de la demande de la charge de travail. 

Tout d'abord, on propose quatre politiques de compensation énergétique, dont l'objectif principal est d'adapter le plan hors ligne pour rapprocher les niveaux de stockage des niveaux prévus à la fin d'une fenêtre temporelle. Ces adaptations sont nécessaires car la production et la demande d'énergie peuvent varier. On montre l'impact du respect des niveaux de stockage sur la qualité de service. Ensuite, on introduit des algorithmes d'apprentissage par renforcement dans le problème de la compensation, afin d'améliorer la qualité de service de nos politiques. En fait, on utilise ces algorithmes pour définir la compensation des différents scénarios, dans le but d'augmenter le nombre de travaux terminés. Ainsi, on pourrait respecter les niveaux de stockage d'énergie tout en améliorant la qualité de service. Enfin, on présente l'heuristique BEASY, qui combine les décisions relatives à la puissance et à l'ordonnancement pour réduire le nombre de tâches abandonnées et le gaspillage d'énergie. Elle utilise des prédictions pour prendre des décisions, en identifiant les moments dangereux pour être plus prudent. BEASY respecte les contraintes de puissance, avec un faible gaspillage d'énergie et le moindre nombre de tâches détruites parmi tous les algorithmes.

    \cleardoublepage
    Doing a PhD abroad was one of the biggest challenges of my life, and I would like to acknowledge everyone who helped me to reach this achievement.

First, I thank my advisors, Patricia Stolf and Stephane Caux, for the lessons, patience, and guidance over these three years. You were essential in every part of this work, always reminding me to balance development and research. Also, I want to express my gratitude to Rodrigo Righi, my master's advisor from Brazil, for all the help before the PhD. All these professors made me the researcher that I am today. 

I thank the reviewers, Jesus Carretero and Jean-Marc Menaud, and the jury members, Pierre-François Dutot, Raphael Feraud, and Bruno Sareni. Your suggestions, questions, and scientific exchanges were essential to this work validation. Regarding my work, I would like to express my gratitude to the Agence Nationale de la Recherche (ANR) for funding my PhD as a part of the DATAZERO2 project (ANR-19-CE25-0016).

I would also like to thank all my colleagues from the DATAZERO2 project: Alexis, Amal, Christophe, Emilien, Georges, Jean-Marc Nicod, Jean-Marc Pierson, Jérôme, Laurent, Leila, Louis-Claude, Millian, Robin, and Véronika. All the discussions with you were important for my development and this thesis. To the DATAZERO2's PhD students, Damien, Daniel, Maël, and Manal, I am very grateful for all the moments that we shared, from the work meetings to the fun times.  

For my colleagues and friends made through the LAPLACE laboratory, especially Abdulrahman, Alessandro Poles, Alessandro Oliveira, Andrea, Carole, Corentin, Daouda, Diego, Dov, Dyalla, Erick, Evelise, Fernanda, Gianmarco, Gregory, Joseph, Lorenzo, Luc, Lucas, Maxime, Miguel, Paul, Pawel, Rafael Accacio, Ryan, and Youssef, my eternal gratitude. You helped me to go through all the challenges imposed by a thesis. Without your support, I do not know if I could finish it. 

I also want to express my gratitude to my family for all your support and encouragement during my life and this PhD. It is hard to be away from you, but I was still able to receive your love. I can not forget to thank my Brazilian friends, especially Samuel and Bruno Fernandes. Your funny messages were what I needed in some difficult moments. Finally, I thank my wife, Carla. You supported me since we met. You bring color to my life, and I love you so much. This thesis is not only mine, it is also yours. Thank you for all your love and caring. 
    \cleardoublepage
    \tableofcontents
    \cleardoublepage
    \listoffigures
    \listoftables
    \cleardoublepage

    \pagenumbering{arabic}
    \chapter{Introduction}
\label{chap1}

\section{Context}

\section{Problem Statement}

\section{Research Goals}

\section{Summary of Contributions}

\section{Publications and Communication}

\section{Dissertation Outline}
The remaining portion of the dissertation is organized as follows:
\begin{itemize}
    \item[] \textbf{Chapter 2 - Related Work and Context:} 
    \item[] \textbf{Chapter 3 - Modelling, Data, and Simulation:} 
    \item[] \textbf{Chapter 4 - Introducing Power Compensations:} 
    \item[] \textbf{Chapter 5 - Learning Power Compensations:} 
    \item[] \textbf{Chapter 6 - Adding Battery Awareness in EASY Backfilling:} 
    \item[] \textbf{Chapter 7 - Middleware integration:} 
    \item[] \textbf{Chapter 8 - Conclusion and Perspectives:} 
\end{itemize}

    \cleardoublepage
    \chapter{Context and Related Work}
\label{cha:related_work}
\minitoc

\section{Global Warming and ICT Role}
Global warming is one of the most critical environmental issues of our day \cite{houghton2005global}. Global warming is the effect of human activities on the climate, mainly the burning of fossil fuels (coal, oil, and gas) and large-scale deforestation \cite{houghton2005global}. Both activities have grown immensely since the industrial revolution. The burning of fossil fuels process results in greenhouse gas emissions \cite{olabi2022renewable}. Today, fossil fuels are one of the world's main sources of energy production, helping to emit more and more GHG \cite{olabi2022renewable}. GHG stays in the atmosphere creating a layer as a blanket over the planet's surface. Without this blanket, the Earth can balance the radiation energy from the sun and the thermal radiation from the Earth to space \cite{houghton2005global}. However, this human-generated blanket imposes a barrier to the thermal radiation from the Earth, letting it into the atmosphere and heating the planet, working as a greenhouse. All this process works as a greenhouse which is the reason for the name greenhouse gas \cite{houghton2005global}.

This situation brings us to United Nations Climate Change Conference (COP21) in Paris, France, on 12 December 2015. At this conference, 196 signed the Paris Agreement aiming to \cite{nations_paris_nodate}:
\begin{enumerate}
    \item Reduce global greenhouse gas emissions substantially, limiting the global temperature increase in this century to 2\degree C while pursuing measures to limit the growth even further to 1.5\degree C;
    \item Review countries’ commitments every five years (through the Nationally Determined Contribution, or NDC);
    \item Provide financing to developing countries to mitigate climate change, strengthen resilience, and enhance their abilities to adapt to climate impacts. 
\end{enumerate}

These are ambitious but necessary objectives. Since then, countries and organizations have proposed several actions and pledges. However, a recent report indicates that the actual world's effort is not enough \cite{tracker2022projections}. Figure \ref{fig:ghg_cat} shows GHG emission and temperature estimations. We could see that there is a small reduction in emissions increase tendency. Nevertheless, this figure estimates that real-world actions based on current policies will lead to an increase of somewhere between 2.6 and 2.9\degree C by 2100. This estimation is well above the 1.5\degree C pursued by the Paris Agreement. Considering the targets proposed by the countries through NDC, the temperature will be around 2.4\degree C. In a scenario based on NDC targets and submitted and binding long-term targets, the prediction is a temperature of 2\degree C by 2100, the limit proposed by the Paris Agreement. The report forecasts an optimistic scenario analyzing the effect of net zero emissions targets of about 140 countries that are adopted or under discussion. Even in this optimistic scenario, the estimated temperature would be 1.8\degree C. The situation tends to be even worst with the gold rush for gas \cite{tracker2022massive}. The report indicates that in 2022 we arrived at 1.2\degree C warming \cite{tracker2022projections}.

\begin{figure}[!htb]
    \centering
    \includegraphics[scale=0.09]{Images/Related_works/Emissions_2022-11.png}
    \caption{Estimated global GHG emissions \cite{tracker2022projections}.}
    \label{fig:ghg_cat}
\end{figure}

We have started to feel the impacts of global warming on humanity, such as heatwaves, droughts, and floods, impacting flora and fauna directly \cite{masson2018global, change2022threat}. In a cascade effect, this increases food and water insecurity worldwide \cite{change2022threat, doi:10.1126/science.1239402}. Also, high temperatures increase mortality, impact labor productivity, impair learning, increase adverse pregnancy outcomes possibility, increase conflict, hate speech, migration, and infectious disease spread \cite{lenton2023quantifying}. Therefore, an increase of the temperature by 2.7\degree C as forecasted would impact one-third (22–39\%) of the world's population by 2100 \cite{lenton2023quantifying}. Climate change has already impacted around 9\% of people (>600 million) \cite{lenton2023quantifying}. Reducing global warming from 2.7 to 1.5\degree C results in a $\sim$5-fold decrease in the population exposed to unprecedented heat (mean annual temperature $\geq$29\degree C) \cite{lenton2023quantifying}. Thus, all sectors must reduce their GHG emissions as much as possible.

Information and Communication Technology is one of these sectors which has accelerated growth in the last 70 years. Unesco defines ICT as \cite{unesco2009guide}:

\begin{quote}
    ``Information and communication technologies (ICT) is defined as a diverse set of technological tools and resources used to transmit, store, create, share or exchange information. These technological tools and resources include computers, the Internet (websites, blogs and emails), live broadcasting technologies (radio, television and webcasting), recorded broadcasting technologies (podcasting, audio and video players, and storage devices) and telephony (fixed or mobile, satellite, visio/video-conferencing, etc.).''
\end{quote}

Regarding the ICT role in GHG emissions, its current global share is around 1.8\%-2.8\%, or 2.1\%-3.9\% considering the supply chain pathways.

% \begin{itemize}
%     \item Present the numbers of global warming generally;
%     \item Present the predictions about the global warming;
%     \item Introduce the role of ICT generally;
%     \item Write about data center impact;
% \end{itemize}

\section{Renewable Energy}
\begin{itemize}
    \item Explain the possible sources of renewable energy;
    \item Show that Renewable energy is a possible way to reduce the global warming problem;
    \item Write about the uncertainties;
\end{itemize}

\section{Renewable-only Data center}
\begin{itemize}
    \item Explain the possibility of applying renewable in data centers;
    \item Show that some big cloud providers are doing it, but not entirely;
    \item Present the challenges in a renewable-only data center;
\end{itemize}

\subsection{Electrical elements}
\begin{itemize}
    \item Write about wind turbines;
    \item Write about solar panel;
    \item Write about battery;
    \item Write about hydrogen;
\end{itemize}

\subsection{IT elements}
\begin{itemize}
    \item Write about the servers;
    \item Write about the power consumption (e.g., DVFS, on-off, idle, etc);
    \item Write about the jobs (e.g., types, resources demanded, etc);
\end{itemize}

\section{Sources of Uncertainty}

\subsection{Weather Uncertainties}
\begin{itemize}
    \item Describe wind uncertainty;
    \item Describe solar irradiation uncertainty;
    \item Describe temperature uncertainty;
\end{itemize}

\subsection{Workload Uncertainties}
\begin{itemize}
    \item Describe job arrival uncertainty;
    \item Describe job size uncertainty;
\end{itemize}

\subsection{Optimization Strategies for Dealing with Uncertainties}
\begin{itemize}
    \item Write about weather predictions
    \item Write about optimization for the weather;
    \item Write about scheduling algorithms to deal with workload uncertainties;
    \item Write about mixing both renewable production and workload uncertainties;
\end{itemize}

\section{Literature Review}

\begin{itemize}
    \item Present the 20 articles selected;
    \item Present a table with each article and the following points:
    \begin{itemize}
        \item Name;
        \item Year;
        \item Source of power (solar, wind, battery, grid, etc);
        \item Level of decision (offline, online, both);
        \item Power adaptations (battery compensations, renewable adaptations, etc).
    \end{itemize}
\end{itemize}

\subsection{Discussion and Classification of the Literature}
    \cleardoublepage
    \chapter{Modelling, Data, and Simulation}

\section{Model}

\subsection{Offline Decision Modules}

\subsubsection{Power Decision Module}

\subsubsection{IT Decision Module}

\subsection{Offline Plan}

\subsection{Online Decision Modules}

\subsubsection{Job scheduling}

\subsubsection{Modifying Power Plan}

\subsubsection{Modifying IT Plan}

\section{Data}

\subsection{Workload Trace}

\subsection{Weather Trace}

\subsection{Platform Configuration}

\section{Simulation}

\subsection{Simulator}

\subsection{Metrics}

\section{Conclusion}
    \cleardoublepage
    % \input{Sections/4-Power_capping.tex}
    % \cleardoublepage
    \chapter{Introducing Power Compensations}
\label{cha:power_compensations}

\minitoc

\section{Introduction}

\section{Model}

\section{Heuristics}

\section{Results Evaluation}

\section{Conclusion}
    \cleardoublepage
    \chapter{Learning Power Compensations}
\label{cha:learning_power_compensations}

\minitoc

This Chapter proposes the introduction of Reinforcement Learning (RL) to choose the best compensation policy. We previously saw that the heuristics proposed (Next, Peak, Last, and Workload) have good results compared to just following the plan. The best heuristic depends on the workload and power production. So, the idea is to let RL algorithms learn which are the best policies to use at each different moment inside the three-day time window. The following sections will describe our approach to solving the compensation problem. First, we start presenting the algorithms used in this section to compare with the previous results. Then, we define the state, action, and rewards. Finally, we present the results and a discussion.

\section{Reinforcement Learning}

As mentioned in Chapter \ref{cha:related_work}, Reinforcement Learning (RL) performs a trial-and-error approach, where an agent explores an environment, takes actions, and receives feedback \cite{kaelbling1996reinforcement}. Three components compose an RL model: \textbf{state}, \textbf{actions}, and \textbf{reward}. Let's exemplify it, by using RL to define the content of a website. A website can apply RL to define which content to display for every user. The idea is to discover the user's subject preferences (e.g., sports, politics, technology, etc.) The \textbf{state} is the information about the user, such as age, previous subjects read, etc. Using this \textbf{state}, the RL evaluates it and chooses the articles to show to the user. The chosen articles' subjects are the \textbf{actions}. Finally, the \textbf{reward} can be 1 if the user clicked on the article and 0 if the user did not click on it. RL algorithm tries to maximize (in this case) the \textbf{reward}, increasing the number of clicks. Since RL does not know the user's behavior, it tries some articles. The clicked articles reinforce the user's subject preferences. Then, the following process is performed at each decision moment (see Figure \ref{fig:reinforcement}):
\begin{enumerate}
    \item RL receives a state from the environment;
    \item RL verifies which action to take for this state;
    \item RL applies the action in the environment;
    \item The environment returns a reward for this action;
    \item RL uses the reward to calculate the relation between state and action;
    \item The environment goes to a new state, restarting the process.
\end{enumerate}

The RL interacts with the environment in this way several times. RL uses the feedback (reward) to learn which are the best actions for the states. Another important aspect is the exploration-exploitation. Since in the early interactions, the RL agent does not know the environment, it starts exploring the different actions in the state. After some interactions, the agent stops the exploration and starts to exploit the actions with higher rewards in the past. Different approaches can be used to model the exploration-exploitation transition, which also depends on the RL algorithm. In this thesis, we used two reinforcement learning algorithms: Q-Learning and Contextual Multi-Armed Bandit. Section \ref{sec:RL_algos} present them. However, before presenting the algorithms, we describe the state, action, and rewards definition. Our objective with RL is to define: \textit{where to compensate the power difference (positive or negative) for each step within a three-day time window?} For example, the best actions in the early steps can be to put the power in the steps with a higher power deficit, but not too much to not dry the battery too fast. So, RL must find a balance between the actions. For ease the comprehension, this chapter uses the following terms:
\begin{itemize}
    \item \textit{Decision step}: The step that takes action to modify the power of a future step. It will receive a reward according to the impact of this action. It can impact only one future step;
    \item \textit{Future step}: The step that receives more or less power from one or several decision steps. When this future step finishes, it is possible to give back a reward for the decision step(s);
    \item \textit{Iterations}: During the learning process, an iteration is the re-execution of an experiment without changing the workload and production but using the previous knowledge;
\end{itemize}

\section{States}

In our problem, the state must englobe three aspects: \textit{moment of the decision}, \textit{how far we are from the plan}, and the \textit{energy to compensate}. The \textit{moment of the decision} is the decision step. For example, for a 3-day time window divided into 5-minutes steps, we have an integer from 0 to 863. The idea behind this aspect is the best decisions in the early steps are different from the best ones in the late steps. For \textit{how far we are from the plan} aspect, we calculate the difference between the actual and predicted state of charge:

\begin{equation}
    \Delta SoC = SoC^{actual}_{t} - SoC^{plan}_{t}
\end{equation}

$\Delta SoC$ indicates if the battery is close to the plan ($\Delta SoC$ close to 0), has less energy than predicted ($\Delta SoC < 0$), or has more energy than predicted ($\Delta SoC > 0$). For example, if the battery has less energy than predicted, it is better to reduce the usage in the next steps. Finally, the \textit{energy to compensate} is given by energy needed to compensate to make $SoC^{plan}_{t}$ at the end of the time window equal to the target. This can be positive (we need to discharge the battery) or negative (we need to charge the battery). This variable can indicate, for example, that a big positive energy compensation must be done in the \emph{Last} step, instead of \emph{Load} (avoiding drying the battery too fast). So, our state is composed by:

\begin{itemize}
    \item Decision step: Integer from 0 to 863;
    \item \(\Delta SoC\): Difference from the actual and planned SoC;
    \item Energy compensation: The power compensation from Equation \ref{equ:energy_battery};
\end{itemize}

\section{Actions}

Since our problem is to define where to compensate the power, our actions are in which future step to compensate. Here, we have a big difference between RL and the previous heuristics. In our heuristics, at each decision step, we will compensate the energy entirely. This means that one decision step can change several future steps. In RL, we will choose only one step, letting the power remain to compensate in the future. Let the future step be $t'$. We define two ways to choose $t'$. The first way is similar to our heuristics. In this way, RL has four actions: Next, Peak, Last, and Workload. Figure~\ref{fig:compensation} shows these heuristics. The second way is to decide at which hour to compensate. This way gives more freedom to RL to define the best step. The hour action includes 72 possible actions (a three-day time window has 72 hours). Therefore, the RL algorithm takes a step inside the hour chosen. To specify exactly which step inside the hour, it applies the peak policy (takes the higher usage to negative compensation and lower usage to positive compensation). Another possibility would be to choose the exact step to compensate. However, this would increase too much the action space (864 possible actions), demanding more time to learn. So, we have the following actions:
\begin{itemize}
    \item Heuristic: We choose next, peak, workload, or last;
    \item Hour: We choose the hour inside the time window.
\end{itemize}

\section{Rewards}

The reward is one of the most important elements in RL because it drives its learning. We have defined three rewards linked to Quality of Service (QoS). We proposed three rewards to verify different possibilities. Mainly, we consider started jobs, finished jobs, and killed jobs, in our rewards. The rewards are: \emph{started jobs}, \emph{finished jobs by step}, and \emph{finished jobs spread}. The \emph{started jobs} reward considers the number of started jobs. The reward given by this action is defined as:
\begin{itemize}
    \item If the future step $t'$ kills at least one job: -1 multiplied by the number of the jobs killed;
    \item If the future step $t'$ were improved (the sum of compensations on step $t'$ is positive) and step $t'$ does not kill any job: 1 multiplied by the number of the jobs started;
\end{itemize}

Since we can have more than one decision step changing the future step $t'$, we must distribute this reward between the decision steps. To do so, we spread the reward according to how much each decision step impacts the future step $t'$. Also, we consider only the decision steps that helped to arrive at the reward. For example, when we have a negative reward, we take only the decision steps that reduce the energy at the step. For example, if we have only 3 decision steps that reduced the energy in future step $t'$ with -1500 Wh, -1000 Wh, and -500 Wh (sum of -3000 Wh), respectively, and the scheduler killed one job in future step $t'$, the reward (-1) will be:
\begin{itemize}
    \item Decision step 1: \(\frac{-1}{3000} \times 1500 = -0.5\)
    \item Decision step 2: \(\frac{-1}{3000} \times 1000 = -0.333333333\)
    \item Decision step 3: \(\frac{-1}{3000} \times 500 = -0.166666667\)
\end{itemize}

The second reward uses the number of finished jobs. This reward gives the reward for all decision steps that impacted the job. It is defined as:
\begin{itemize}
    \item If a job is killed: -1;
    \item If a job is finished: 1;
\end{itemize}

We distribute the reward between the decision steps in the same way as the previous reward. The only difference here is that we consider all steps that the job passed by. For example, if a job is killed, we give a proportion of -1 for each decision step that decreases the power in the steps that the job was running, according to how much it impacted. This approach aims to reinforce the decisions that help to finish jobs and discourage the ones that kill jobs. The last reward is quite similar to the previous one, but in this new one, every time we kill a job, we spread -1 for all previous steps. So, we divide -1 by the number of steps and give the same reward portion value for every step. Our idea here is to solve the problem of the RL avoiding steps with a high number of killed jobs (we present this problem later).

Since the environment can calculate the reward only after executing the future step (to know how many started, finished, and killed jobs in the step $t'$), the RL algorithm will not receive the reward right after the decisions. For example, in iteration 0, the RL algorithm does not have prior knowledge, choosing only random actions. We standardized the reward update at the end of the iteration, reducing the bias (e.g., giving a reward earlier for \emph{Next} action than \emph{Last} can make \emph{Next} be chosen more times). So, at the end of iteration 0, we calculate all rewards for all actions in this iteration. Then, iteration 1 uses the knowledge from iteration 0 in the decision-making process, updating the reward at the end of iteration 1.

\section{Algorithms}
\label{sec:RL_algos}
In this section, we present the algorithms used in this section. We compare the following algorithms with the baselines and heuristics from the previous chapter.

\subsection{Random}
Before describing the RL algorithms, we present two random algorithms. We used these random algorithms to verify if the RL results are because the algorithm learned the best actions or if it is a result of randomness. Since we have two possible actions (heuristic and hour), we also proposed two random algorithms. The first one is \emph{Random heuristic}, which chooses randomly a heuristic (\emph{Peak}, \emph{Next}, \emph{Last}, or \emph{Load}). The second heuristic is named \emph{Random hour}, choosing a random hour. Both random algorithms compensate only in one future step, in the same way as the RL.

\subsection{Q-Learning}
Q-Learning can be understood as a table where the rows are the state, the columns are the actions, and the values are the Q-value. Therefore, for each row (state) and column (action), a Q-value is calculated using the Bellman equation:

\begin{equation}
    Q^{new}(S_t, a_t) = (1 - \alpha)\ Q(S_t, a_t) + \alpha (r_t + \gamma \max_a Q(S_{t+1}, a) )
\end{equation}

Where:
\begin{itemize}
    \item $Q^{new}(S_t, a_t)$: New Q-value;
    \item $S_t$: State;
    \item $a_t$: Action;
    \item $Q(S_t, a_t)$: Actual Q-value;
    \item $\alpha$: Learning factor;
    \item $r_t$: Reward observed;
    \item $\gamma$: Discount factor;
    \item $S_{t+1}$: New state after taking action $a_t$ at state $S_t$;
    \item $\max_a Q(S_{t+1}, a)$: Best expected future reward.
\end{itemize}

For every action taken in a state, Q-Learning updates the Q-value using this equation. $Q(S_t, a_t)$ is the Q-Value before the action taken. $\alpha$ indicates how the new information overrides the old information. $\alpha = 0$ makes the agent exploits prior knowledge exclusively, while $\alpha = 1$ makes the agent consider only the most recent information. $\gamma$ determines the importance of the expected future reward. $\gamma = 0$ makes the agent ignore future rewards, while $\gamma = 1$ makes the agent makes it attempt for a long-term high reward. As presented before, we calculate the reward only at the end of the iteration. Therefore, at the end of each iteration, we use the Bellman equation to update the Q-values of the actions taken in this iteration. Since we updated the reward in the end, we know the state transition (from $S_{t}$ to $S_{t+1}$).

A Q-Learning limitation is that both state and action must be discrete because Q-Learning uses a table and having state or action as continuous would demand an infinite table. Hence, we simplified some of the state variables (action is discrete already). The Decision step variable is discrete (from 0 to 863), so there is no need to change it. Considering the Power compensation, we calculate the percentage of the power compensation according to the battery size, splitting the result into slices of 10\% (from -100\% to 100\%). For example, if the compensation is 140000 Wh and the battery size is 800000 Wh, the compensation is 17.5\% of the battery, which puts it in the slice between 10\% and 19.99999\%. Each slice has an index, discretizing this value. We did the same for the delta SoC variable (also from -100\% to 100\%).

To model the exploration-exploitation policy, we use the epsilon-greedy policy (or $\epsilon-greedy$). We start with $\epsilon = 1$ and we reduce it for every new iteration. Using $\epsilon$, we verify if we take the higher $Q(S_t, a_t)$ or a random action:

\begin{equation}
    a_t = \begin{cases}
        max Q(S_t, a_t), & with\ probability\ 1 - \epsilon \\
        random\ a_t, & with\ probability\ \epsilon \\
    \end{cases}
\end{equation}

So, iteration 0 has $\epsilon = 1$, choosing random actions for every state. At the end of every iteration, we reduce $\epsilon$ by $0.0125$. For example, iteration 1 will have $\epsilon = 0.9875$, allowing it to use a little from prior knowledge. $\epsilon$ will be $0.975$ in iteration 2, $0.9625$ in iteration 3, and so on. We define the learning factor $\alpha = 0.1$ and discount factor $\gamma = 0.9$. These were the values with better results in our different tests.

\subsection{Contextual Multi-Armed Bandit with LinUCB}

While Q-Learning works with a table creating the relationship between states, actions, and rewards, Contextual Multi-Armed Bandit (especially using the LinUCB algorithm) tries to find a linear relation between them \cite{li2010contextual}. LinUCB (Linear Upper Confidence Bound) algorithm learns the correlation between the state (named context in Multi-Armed Bandit) and the reward, using linear regression. This algorithm estimates the upper confidence bound, using the standard deviation of previous rewards. So, the algorithm chooses the arm (action) with higher upper confidence bound, taking the arm with the higher possible return. Figure \ref{fig:linucb} illustrates this behavior, where even with action (arm) 3 having a higher average, it chooses action 2 because of the higher UCB.

\begin{figure}[!htb]
    \centering
    \includegraphics[scale=0.58]{Images/Learning_compensations/linucb.jpg}
    \caption{LinUCB algorithm for choosing the best arm \cite{recommender2020}.}
    \label{fig:linucb}
\end{figure}

Algorithm \ref{alg:linucb} shows the LinUCB algorithm. Let $d$ be the number of variables of the context $x_{t,a}$ (state). In our experiments, $d=3$. $A_a$ and $b_a$ are the variables used in ridge regression. This regression tries to find the correlation between state and reward. Lines 2-5 initialize both $A_a$ and $b_a$. This initialization makes all arms (actions) start with a very high UCB, forcing each arm to be chosen at least once. So, for each step t (line 6), it calculates the UCB for every arm (lines 7-10). To do so, line 8 calculates the ridge regression ($\hat{\theta_a}$), and line 9 applies it to find the UCB. The UCB $\rho_{t, a}$ is calculated by $\rho_{t, a} \leftarrow \hat{\theta_a}^\top x_{t,a} + \nu \sqrt{x_{t,a}^\top A_{a}^{-1} x_{t,a}}$, where the first part ($\hat{\theta_a}^\top x_{t,a}$) is the expected mean, and the second part ($\nu \sqrt{x_{t,a}^\top A_{a}^{-1} x_{t,a}}$) is the upper confidence bound. $\nu$ is a hyperparameter to indicate the importance of the standard deviation in the UCB. The higher $\nu$ is, the wider the confidence bounds become. So, a higher $\nu$ results in a higher emphasis placed on exploration instead of exploitation. We defined $\nu = 20$ after some experiments, with the best results with this value.

\IncMargin{1em}
\begin{algorithm}[!htb]
    \footnotesize
    \SetAlgoLined
    \Begin{
        \ForAll{a $\in A_t$ }{
            $A_a \leftarrow I_d$ (d-dimensional identity matrix)\;
            $b_a \leftarrow 0_{dx1}$ (d-dimensional zero vector)\;
        }        
        \For{t = 0, 1, 2, 3, ..., T}{
            \ForAll{a $\in A_t$ }{
                $\hat{\theta_a} \leftarrow A_{a}^{-1}b_a$\;
                $\rho_{t, a} \leftarrow \hat{\theta_a}^\top x_{t,a} + \nu \sqrt{x_{t,a}^\top A_{a}^{-1} x_{t,a}}$\;
            }
            Choose arm $a_t = \arg\max_{a \in A_t} \rho_{t, a}$, and observe a real-valued payoff $r_t$\; 
            $A_{a_{t}} \leftarrow A_{a_{t}} + x_{t,a_{t}} x_{t,a_{t}}^\top$\;
            $b_{a_{t}} \leftarrow b_{a_{t}} + r_t x_{t,a_{t}}$\;
        }
    }
    \caption{LinUCB algorithm \cite{li2010contextual}.}
    \label{alg:linucb}
\end{algorithm}
\DecMargin{1em}

Differently from Q-Learning, Contextual Multi-Armed Bandit does not need a discretization of the state. Therefore, we can use directly the state here, without modifications.

\section{Results Evaluation}

After presenting the algorithms, we apply them to the critical scenarios from the previous chapter. We have four different RL executions combining the RL algorithm and action: 

\begin{enumerate}
    \item Bandit + heuristic;
    \item Q-Learning + heuristic;
    \item Bandit + hour;
    \item Q-Learning + hour;
\end{enumerate}

These algorithms use the same scheduling from Chapter \ref{cha:power_compensations}, changing only in which step compensating. For each critical case, we run 200 iterations of each RL algorithm with the different reward types. The idea is to verify if the RL algorithms can learn by repeating the same inputs (workload and weather). Also, we want to verify if they can improve the number of finished jobs. We do not execute the average cases due to their complexity. They are 100 different cases, demanding high processing time for executing 200 iterations of each one. 

\subsection{Started Jobs Reward}

\begin{figure}[!htb]
    \centering
    \includegraphics[scale=0.29]{Images/Learning_compensations/reward_started_profile_best_workload_1_with_noise_state_delta.pdf}
    \caption{Results of reward started jobs in critical case 1.}
    \label{fig:started_reward_results_critical_1}
\end{figure}

\begin{figure}[!htb]
    \centering
    \includegraphics[scale=0.29]{Images/Learning_compensations/reward_started_profile_best_workload_2_with_noise_state_delta.pdf}
    \caption{Results of reward started jobs in critical case 2.}
    \label{fig:started_reward_results_critical_2}
\end{figure}

\begin{figure}[!htb]
    \centering
    \includegraphics[scale=0.29]{Images/Learning_compensations/reward_started_profile_worst_workload_1_with_noise_state_delta.pdf}
    \caption{Results of reward started jobs in critical case 3.}
    \label{fig:started_reward_results_critical_3}
\end{figure}

\begin{figure}[!htb]
    \centering
    \includegraphics[scale=0.29]{Images/Learning_compensations/reward_started_profile_worst_workload_2_with_noise_state_delta.pdf}
    \caption{Results of reward started jobs in critical case 4.}
    \label{fig:started_reward_results_critical_4}
\end{figure}

\clearpage

\subsection{Finished Jobs by Step Reward}

\begin{figure}[!htb]
    \centering
    \includegraphics[scale=0.29]{Images/Learning_compensations/reward_finished_touched_profile_best_workload_1_with_noise_state_delta.pdf}
    \caption{Results of reward finished jobs by step reward in critical case 1.}
    \label{fig:touched_reward_results_critical_1}
\end{figure}

\begin{figure}[!htb]
    \centering
    \includegraphics[scale=0.29]{Images/Learning_compensations/reward_finished_touched_profile_best_workload_2_with_noise_state_delta.pdf}
    \caption{Results of reward finished jobs by step reward in critical case 2.}
    \label{fig:touched_reward_results_critical_2}
\end{figure}

\begin{figure}[!htb]
    \centering
    \includegraphics[scale=0.29]{Images/Learning_compensations/reward_finished_touched_profile_worst_workload_1_with_noise_state_delta.pdf}
    \caption{Results of reward finished jobs by step reward in critical case 3.}
    \label{fig:touched_reward_results_critical_3}
\end{figure}

\begin{figure}[!htb]
    \centering
    \includegraphics[scale=0.29]{Images/Learning_compensations/reward_finished_touched_profile_worst_workload_2_with_noise_state_delta.pdf}
    \caption{Results of reward finished jobs by step reward in critical case 4.}
    \label{fig:touched_reward_results_critical_4}
\end{figure}

\clearpage

\subsection{Finished Jobs Spread Reward}

\begin{figure}[!htb]
    \centering
    \includegraphics[scale=0.29]{Images/Learning_compensations/reward_finished_spread_profile_best_workload_1_with_noise_state_delta.pdf}
    \caption{Results of reward finished jobs spread reward in critical case 1.}
    \label{fig:spread_reward_results_critical_1}
\end{figure}

\begin{figure}[!htb]
    \centering
    \includegraphics[scale=0.29]{Images/Learning_compensations/reward_finished_spread_profile_best_workload_2_with_noise_state_delta.pdf}
    \caption{Results of reward finished jobs spread reward in critical case 2.}
    \label{fig:spread_reward_results_critical_2}
\end{figure}

\begin{figure}[!htb]
    \centering
    \includegraphics[scale=0.29]{Images/Learning_compensations/reward_finished_spread_profile_worst_workload_1_with_noise_state_delta.pdf}
    \caption{Results of reward finished jobs spread reward in critical case 3.}
    \label{fig:spread_reward_results_critical_3}
\end{figure}

\begin{figure}[!htb]
    \centering
    \includegraphics[scale=0.29]{Images/Learning_compensations/reward_finished_spread_profile_worst_workload_2_with_noise_state_delta.pdf}
    \caption{Results of reward finished jobs spread reward in critical case 4.}
    \label{fig:spread_reward_results_critical_4}
\end{figure}

\clearpage

\subsection{Discussion}

\section{Conclusion}
    \cleardoublepage
    \chapter{Adding Battery Awareness in EASY Backfilling}
\label{cha:heuristic}

\section{Introduction}

\section{Model}

\section{Heuristic}

\subsection{Predictions}

\subsection{Job Scheduling}

\subsection{Power compensation}

\section{Conclusion}
    \cleardoublepage
    % \chapter{Middleware Integration}

\section{Introduction}

\section{Frameworks}

\section{Containerization}

\section{Conclusion}
    % \cleardoublepage
    \chapter{Conclusion and Perspectives}
\label{cha:conclusion}

\section{Conclusion}

\section{Perspectives}
    \cleardoublepage

    \bibliographystyle{IEEEtran}
    \bibliography{library}

\end{document}
